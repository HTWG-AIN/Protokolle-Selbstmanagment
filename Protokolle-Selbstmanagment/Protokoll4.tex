\documentclass[11pt,a4paper]{article}
\usepackage[utf8]{inputenc}
\usepackage{amsmath}
\usepackage{amssymb}
\usepackage{graphicx} 
\usepackage[german]{babel}
 


\begin{document}
%%%%%%%%%%%%%%%%%%%%%%%%%%%%%%%%%%%%%%%%%%%%%%%%%%%%%%%%%%%%%%%%%%%
	\begin{figure}
  	\hspace*{-15.0mm} {\includegraphics[width=50mm]{logo}}
  	\end{figure}
  
\hspace{10mm}
%%%%%%%%%%%%%%%%%%%%%%%%%%%%%%%%%%%%%%%%%%%%%%%%%%%%%%%%%%%%%%%%%%%
\begin{center}

\textit{\textbf{\Huge{ Studienmethodik und Selbstmanagement  SS 2015}}}\\
\emph{\textbf{4.Protokoll Gruppe 1} }
\begin{flushleft}
\textit{ Dozentin :  Antje Grießmayer }
\end{flushleft}

\end{center}
%%%%%%%%%%%%%%%%%%%%%%%%%%
\begin{flushright}
vom 18.05.2015
\end{flushright}
%%%%%%%%%%%%%%%%%%%%%%%%%%
\section*{Modulstermine}
\begin{enumerate}
\item Modul 23.03.15
\item Modul 20.04.15
\item Modul 04.05.15
\emph{\textbf{\item Modul 18.05.15}}

\item Prüfung 15.06.15 um 11:00
\end{enumerate}
\section*{Paretoprinzip :  \textit{Pareto-Effekt, 80-zu-20-Regel}}
besagt, dass 80 \% der Ergebnisse in 20 \% der Gesamtzeit eines Projekts erreicht werden. Die verbleibenden 20 \% der Ergebnisse benötigen 80 \% der Gesamtzeit und verursachen die meiste Arbeit.

\section*{Studienmethodik}
\begin{itemize}
\item Lernen zu lernen
\item Wie man lernen kann.
\item Mnemotechnik (\emph{Gedächtnistraining}) :Verbesserung des Speicherns und Behaltens von Informationen im Langzeitgedächtnis
\item Zeit koordinieren 
\item Paretoprinzip anwenden oder brauche ich 100 \% ?!
\end{itemize}


\section*{Beziehungskonto}
\begin{tabular}{|l || l|} 
\hline \hline
+ Zuhören & - Lügen\\
+ Mitmachen & - Beschimpfen\\
\vdots & \vdots\\
$\rightarrow$ Wertschätzung & $\rightarrow$ Geringschätzung\\
\hline

\end{tabular}
\section*{Bring -und Holschuld}
Bei Bringschuld ist Leistungs- und Erfüllungsort der Wohnsitz des Gläubigers der Leistung, bei einer Holschuld ist Leistungs- und Erfüllungsort der Wohnsitz des Schuldners

\section*{Email Regeln}
\begin{itemize}
\item Aufm Punkt kommen
\item Möglichst knapp und kurz 
\item Betreff deutlich angeben
\item Cc Kopie an wen
\item Verantwortung geben
\item Anhänge : lesbares Format (PDF)

\end{itemize}
\section*{Gewaltfreie Kommunikation nach \textit{Rosenberg}}
\begin{enumerate}
\item Mit Fakt anfangen
\item Gefühle : bin entäuscht (ICH-Botschaft)
\item Bedürfnisse : Verlsslichkeit ist wichtig sowie Punktlichkeit
\item Bitte : jdn. bitten etwas zu tun (Aus Höfflichkeit)
\end{enumerate}
\section*{Wollen -und Sollen Ziel}
Wollenziel : \textit{intrinsisch} \\
Sollenziel : \textit{extrinsisch}


\end{document}

