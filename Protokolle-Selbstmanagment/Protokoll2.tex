\documentclass[11pt,a4paper]{article}
\usepackage[utf8]{inputenc}
\usepackage{amsmath}
\usepackage{amssymb}
\usepackage{graphicx} 
\usepackage[german]{babel}
 


\begin{document}
%%%%%%%%%%%%%%%%%%%%%%%%%%%%%%%%%%%%%%%%%%%%%%%%%%%%%%%%%%%%%%%%%%%
	\begin{figure}
  	\hspace*{-15.0mm} {\includegraphics[width=50mm]{logo}}
  	\end{figure}
  
\hspace{10mm}
%%%%%%%%%%%%%%%%%%%%%%%%%%%%%%%%%%%%%%%%%%%%%%%%%%%%%%%%%%%%%%%%%%%
\begin{center}

\textit{\textbf{\Huge{ Studienmethodik und Selbstmanagement  SS 2015}}}\\
\emph{\textbf{1.Protokoll Gruppe 1} }
\begin{flushleft}
\textit{ Dozentin :  Antje Grießmayer }
\end{flushleft}

\end{center}
%%%%%%%%%%%%%%%%%%%%%%%%%%
\begin{flushright}
vom 20.04.2015
\end{flushright}
%%%%%%%%%%%%%%%%%%%%%%%%%%
\section*{Modulstermine}
\begin{enumerate}
\item Modul 23.03.15
\emph{\textbf{\item Modul 20.04.15}}
\item Modul 04.05.15
\item Modul 18.05.15
\item Prüfung 15.06.15 um 11:00
\end{enumerate}

\section*{Referat}

\begin{itemize}
\item 4. Weg: Gewinn/Gewinn denken  (Vorträger JOHANNES)
\item 5. Weg: Erst verstehen, dann verstanden werden (Vorträger MARTIN)
\item 6. Weg: Synergien schaffen (Vorträger VINCENT)
\end{itemize}

\section*{Feedbackregeln}
\begin{itemize}
\item Negativ bzw. positive das Referat beurteilen
\item Mit positiv beginnen dann negativ
\item Verbesserungen vorschlagen
\item Keine Verallgemeinerung (Was nicht gut an den Folien)
\item Ich -Botschaften formulieren 
\item konstruktiv kritisieren und ehrlich sein
\item Danke sagt der Feedbackempfänger
\end{itemize}

\section*{Zeitmanagment}
\begin{enumerate}
\item Erste Generation : \\ TO-DO Listen, Checklisten, Zeitplan, Wochenpläne und Kalender
\item Zweite Generation : \\ ( Eisenhower) Prioritätssätze : Verschiedene Arten von Prioritäten 
\item Dritte Generation : \\ Wahrnehmung einer persönlichen Verantwortung im Einklang mit seinen Werten und Zielen
\item Vierte Generation : \\ Stärkung der Bezieungsebene sowie Verbesserung der Lebensqualität (lächeln, Ewartungen, ausgleichen ..)
\end{enumerate}

\section*{Zieldefinition \tiny{wichtig für die Klausur} } 
\textit{SMART+}
\begin{itemize}
\item \textbf{S}\textit{pezifisch}
\item \textbf{M}\textit{essbar}
\item \textbf{A}\textit{nspruchsvoll}
\item \textbf{R}\textit{ealistisch} (etwas konkretes schaffen)
\item \textbf{T}\textit{erminiert} (genaues Datum)
\item \textbf{+} \textit{positiv} formuliert \\ z.B : Denk nicht an Streit ! $\rightarrow$ Das Gehirn kann nicht \emph{NICHT} denken, das ist das sog. \emph{EIFFELTURM-Prinzip!}
\end{itemize}

\section*{Unterscheidung der Zielarten nach Zeit ! \tiny{Futur II benutzen!}}
\begin{enumerate}
\item kurrzfristig ($<$ 3 Jahre) [beruflich, privat]
\item mittelfristig (3-7 Jahre) [beruflich, privat]
\item langfristig ($>$ 7 Jahre) [beruflich, privat]
\end{enumerate}
Z.B : Ich bin am 30.08.2018 Beachelor der AIN !!
\begin{tiny}
 \tiny{Datum immer angeben!} 
\end{tiny}\\

 \section*{Unterscheidung nach Motivation ! \tiny{innere und äußere Motivation} }
 \begin{enumerate}
 \item \textit{intrinsisch} : ezieht sich auf einen Zustand, bei dem wegen eines inneren Anreizes, der in der Tätigkeit selbst liegt.gehandelt wird.
 \item \textit{extrinsisch} :bezieht sich auf einen Zustand, bei dem wegen äußerer Gründe, d.h. wegen der Konsequenzen der Handlungsergebnisse (z.B. positive Personalbeurteilung, Gehaltssteigerung etc.), gehandelt wird.
 \end{enumerate}

\section*{Aufgabe}
Wochenplan überlegen !
\end{document}
